\documentclass[
	pdftex,             % PDFTex verwenden
	a4paper,            % A4 Papier
	oneside,            % Einseitig
	bibtotoc,           % Literaturverzeichnis einfügen bibtotocnumbered: nummeriert
	liststotoc,         % Verzeichnisse einbinden in toc
	idxtotoc,           % Index ins Verzeichnis einfügen
	halfparskip,        % Europäischer Satz mit abstand zwischen Absätzen
	chapterprefix,      % Kapitel anschreiben als Kapitel
	headsepline,        % Linie nach Kopfzeile
	%footsepline,       % Linie vor Fusszeile
	%pointlessnumbers,  % Nummern ohne abschließenden Punkt
	12pt                % Grössere Schrift, besser lesbar am bildschrim
]{scrbook}

%
% Randabstände einstellen
%
\usepackage[top=2cm, bottom=2cm, left=2cm, right=2cm]{geometry}

%
% Paket für Übersetzungen ins Deutsche
%
\usepackage[ngerman]{babel}

%
% Pakete um UTF8 Zeichnensätze verwenden zu können und die dazu
% passenden Schriften.
%
\usepackage[utf8]{inputenc}
\usepackage[T1]{fontenc}

%
% Paket für Quotes
%
\usepackage[babel,german=quotes]{csquotes}

%
% Paket zum Erweitern der Tabelleneigenschaften
%
\usepackage{array}

%
% Paket für schönere Tabellen
%
\usepackage{booktabs}

%
% Paket um Grafiken einbetten zu können
%
\usepackage{graphicx}

%
% Spezielle Schrift im Koma-Script setzen.
%
\setkomafont{sectioning}{\normalfont\bfseries}
\setkomafont{captionlabel}{\normalfont\bfseries} 
\setkomafont{pagehead}{\normalfont\bfseries} % Kopfzeilenschrift
\setkomafont{descriptionlabel}{\normalfont\bfseries}

%
% Zeilenumbruch bei Bildbeschreibungen
%
\setcapindent{1em}

%
% Kopf und Fußzeilen
%
\usepackage{scrpage2}
\pagestyle{scrheadings}
% Inhalt bis Section rechts und Chapter links
\automark[section]{chapter}
% Mitte: leer
\chead{}

%
% Mathematische Symbole aus dem AMS Paket.
%
\usepackage{amsmath}
\usepackage{amssymb}

%
% Type 1 Fonts für bessere Darstellung in PDF verwenden.
%
%\usepackage{mathptmx}           % Times + passende Mathefonts
%\usepackage[scaled=.92]{helvet} % skalierte Helvetica als \sfdefault
\usepackage{courier}            % Courier als \ttdefault

%
% Paket um Textteile drehen zu können
%
\usepackage{rotating}

%
% Paket für Farben im PDF
%
\usepackage{color}

%
% Paket für Links innerhalb des PDF Dokuments
%
\definecolor{LinkColor}{rgb}{0,0,0.5}
\usepackage[%
	pdftitle={Titel},                                         % Titel der Diplomarbeit
	pdfauthor={Autor},                                        % Autor(en)
	pdfcreator={LaTeX, LaTeX with hyperref and KOMA-Script} , % Genutzte Programme
	pdfsubject={Betreff},                                     % Betreff
	pdfkeywords={Keywords}]{hyperref}                         % Keywords halt :-)
\hypersetup{colorlinks=true,                                      % Definition der Links im PDF File
	linkcolor=LinkColor,                                      % Alle Links haben die gleiche oben definierte Farbe
	citecolor=LinkColor,                                      % s.o.
	filecolor=LinkColor,                                      % s.o.
	menucolor=LinkColor,                                      % s.o.
	pagecolor=LinkColor,                                      % s.o.
	urlcolor=LinkColor}                                       % s.o.

%
% Paket um Listings sauber zu formatieren
%
\usepackage[savemem]{listings}
\lstloadlanguages{TeX}

%
% Listing Definitionen für LaTeX Code
%
\definecolor{lbcolor}{rgb}{0.85,0.85,0.85}
\lstset{language=[LaTeX]TeX,
	numbers=left,
	stepnumber=1,
	numbersep=5pt,
	numberstyle=\tiny,
	breaklines=true,
	breakautoindent=true,
	postbreak=\space,
	tabsize=2,
	basicstyle=\ttfamily\footnotesize,
	showspaces=false,
	showstringspaces=false,
	extendedchars=true,
	backgroundcolor=\color{lbcolor}}
%
% ---------------------------------------------------------------------------
%

%
% Neue Umgebungen
%
\newenvironment{ListChanges}%
	{\begin{list}{$\diamondsuit$}{}}%
	{\end{list}}

%
% aller Bilder werden im Unterverzeichnis 'bilder' gesucht
%
\graphicspath{{bilder/}}

%
% Literaturverzeichnis-Stil
%
\bibliographystyle{plain}

%
% Strukturiertiefe bis subsubsection{} möglich
%
\setcounter{secnumdepth}{3}

%
% Dargestellte Strukturiertiefe im Inhaltsverzeichnis
%
\setcounter{tocdepth}{3}

%
% Zeilenabstand wird um den Faktor 1.5 verändert
%
%\renewcommand{\baselinestretch}{1.5}

%
% Abkürzungsverzeichnis
%
\usepackage[intoc]{nomencl}
\usepackage[toc]{glossaries}
% Deutsche Überschrift
\printglossary[title=Abkürzungsverzeichnis]
%\renewcommand{\nomname}{Abkürzungsverzeichnis}
% Punkte zw. Abkürzung und Erklärung
%\setlength{\nomlabelwidth}{.20\hsize}
%\renewcommand{\nomlabel}[1]{#1 \dotfill}
% Zeilenabstände verkleinern
%\setlength{\nomitemsep}{-\parsep}
%\makenomenclature

%
% Kapitel-Überschriften werden in eine Zeile geschrieben.
%
% normal: 
% Kapitel 1.
% Einleitung
%
% wenn hier auskommentiert:
% Kapitel 1: Einleitung
%
%\usepackage{titlesec}
%\titleformat{\chapter}[hang] 
%{\normalfont\huge\bfseries}{\chaptertitlename\ \thechapter:}{1em}{} 

